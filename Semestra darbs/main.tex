\documentclass[a4paper,10pt]{article}

\usepackage{amssymb}
\usepackage{amsmath}
\usepackage[
    a4paper,
    margin=1in,
]{geometry}
\usepackage{graphicx}
\usepackage{xcolor}
\usepackage{fontspec} % Choosing fonts in XeLaTeX/LuaLaTeX
\usepackage{polyglossia} % Language support for XeLaTeX/LuaLaTeX
\usepackage[
    tableposition=top,
    figureposition=bottom,
]{caption}
\usepackage{siunitx}
\usepackage{multirow}
\usepackage{float}
\usepackage{hyperref}
\usepackage{listings}
\usepackage[
  backend=biber,
  style=phys, % or choose a different style (e.g., numeric, authoryear)
]{biblatex}

\setmainlanguage{latvian}
\setotherlanguage{english}
\setmainfont{Latin Modern Roman}
\setsansfont[Scale=MatchLowercase]{Fira Sans}
\setmonofont[Scale=MatchLowercase]{JetBrains Mono}
\frenchspacing
\addbibresource{references.bib}

\renewcommand{\lstlistingname}{Fragments}
\lstset{
    basicstyle=\small\ttfamily,
    keywordstyle=\color{red},
    identifierstyle=\color{blue},
    commentstyle=\color{black!50},
    stringstyle=\color{cyan!50!black},
    backgroundcolor=\color{black!5},
    showstringspaces=false,
    columns=fullflexible,
}

\title{
    Semestra darba tēma\\[-0.2\baselineskip]
    {\large DatZB011: Datori un programmēšana}
}
\author{
  Lars Feynman\\
  (\texttt{lf24012})
  \and
  Linda Freiburg\\
  (\texttt{lf24345})
}

\begin{document}
\maketitle

\section{Ievads}
\dots

\section{Programmas dizains}

\dots

\section{Programmas izstrādes process}
\dots

\section{Programmas testēšana}
\dots

\section{Rezultāti}
\dots

\section{Secinājumi}
\dots

\section{Atsauces}

% \printbibliography

\pagebreak
\section{Pielikums: programmas teksts}

% Šī sadaļa paredzēta atsevišķu NOZĪMĪGU koda fragmentu ievietošanai, ja to apjoms pārsniedz pusi lappuses. 
%Pilnu programmu NAV nepieciešams ievietot.
% Neievietojiet koda fragmentus ekrānšāviņu veidā!
% Vairāk iespējams izlasīt: https://www.overleaf.com/learn/latex/Code_listing

\begin{lstlisting}[language=Python, caption=Python example]
import pandas as pd
import matplotlib.pyplot as plt

# Assuming your .DAT file is tab-separated
dat_file_path = 'bulk_223_rutile_525nm/RAMSPEC.DAT'

# Read the .DAT file into a pandas DataFrame
df = pd.read_csv(dat_file_path, sep='\s+', header=None, engine='python')

# Assuming your .DAT file has 10 columns, you can name them as needed
# Replace 'Column1', 'Column2', ..., 'Column10' with actual column names
df.columns = [
    'wavenumber cm-1',
    'Raman intensity',
    'parallel polarization',
    'perpendicular polarization',
    'xx', 'xy', 'xz', 'yy', 'yz', 'zz'
]

# Specify the path where you want to save the .csv file
csv_file_path = 'RAMSPEC-rutile-bulk-223-525nm.csv'

# Export the DataFrame to a .csv file
df.to_csv(csv_file_path, index=False)

plt.plot(df['wavenumber cm-1'], df['Raman intensity'])
plt.xlabel('Wavenumber (cm$^{-1}$)')
plt.ylabel('Raman intensity')
plt.title('Raman spectrum rutile 2x2x3 bulk (525nm, 298.15K)')
plt.yticks([], [])
plt.show()
\end{lstlisting}

% \section{Pielikums: testa scenāriji} 
% Šajā sadaļā var ievietot testa scenārijus tabulas veidā.
% ...

%Atcerieties, ka pielikumos jāievieto visas lielapjoma tabulas un attāli.

\end{document}
